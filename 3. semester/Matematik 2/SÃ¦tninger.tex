% Options for packages loaded elsewhere
\PassOptionsToPackage{unicode}{hyperref}
\PassOptionsToPackage{hyphens}{url}
%
\documentclass[
]{article}
\usepackage{amsmath,amssymb}
\usepackage{iftex}
\ifPDFTeX
  \usepackage[T1]{fontenc}
  \usepackage[utf8]{inputenc}
  \usepackage{textcomp} % provide euro and other symbols
\else % if luatex or xetex
  \usepackage{unicode-math} % this also loads fontspec
  \defaultfontfeatures{Scale=MatchLowercase}
  \defaultfontfeatures[\rmfamily]{Ligatures=TeX,Scale=1}
\fi
\usepackage{lmodern}
\ifPDFTeX\else
  % xetex/luatex font selection
\fi
% Use upquote if available, for straight quotes in verbatim environments
\IfFileExists{upquote.sty}{\usepackage{upquote}}{}
\IfFileExists{microtype.sty}{% use microtype if available
  \usepackage[]{microtype}
  \UseMicrotypeSet[protrusion]{basicmath} % disable protrusion for tt fonts
}{}
\makeatletter
\@ifundefined{KOMAClassName}{% if non-KOMA class
  \IfFileExists{parskip.sty}{%
    \usepackage{parskip}
  }{% else
    \setlength{\parindent}{0pt}
    \setlength{\parskip}{6pt plus 2pt minus 1pt}}
}{% if KOMA class
  \KOMAoptions{parskip=half}}
\makeatother
\usepackage{xcolor}
\setlength{\emergencystretch}{3em} % prevent overfull lines
\providecommand{\tightlist}{%
  \setlength{\itemsep}{0pt}\setlength{\parskip}{0pt}}
\setcounter{secnumdepth}{-\maxdimen} % remove section numbering
\ifLuaTeX
  \usepackage{selnolig}  % disable illegal ligatures
\fi
\IfFileExists{bookmark.sty}{\usepackage{bookmark}}{\usepackage{hyperref}}
\IfFileExists{xurl.sty}{\usepackage{xurl}}{} % add URL line breaks if available
\urlstyle{same}
\hypersetup{
  pdftitle={Sætninger},
  hidelinks,
  pdfcreator={LaTeX via pandoc}}

\title{Sætninger}
\author{}
\date{}

\begin{document}
\maketitle

\subsection{Differentialligninger}\label{differentialligninger}

\paragraph{Form:}\label{form}

\subparagraph{Homogen}\label{homogen}

hvor {}.

\subparagraph{Inhomogen}\label{inhomogen}

hvor {} og {} er en kontinuert funktion med reelle el. komplekse
værdier.

\paragraph{Løsninger:}\label{luxf8sninger}

Karakteristiske polynomium

1.14 Lin. uaf. løsninger udfra {}

Lad {} være en rod i karakterligningen med am = {}. Da har den homogene
differentialligning de lineært uafhængige løsninger

1.8 Reelle løsninger

Hvis {} er løsning til (H) er også {} og {} løsninger.\\
Dsv, givet {}, {}

1.20 Struktursætningen

Den fuldstændige løsning til en inhomogent differentialligning er

hvor {} er en partikulær løsning til den inhomogene differentialligning,
og {} er den fuldstændige løsning til den tilsvarende homogene
differentialligning.

\#\#\#\# Overføringsfunktionen:

Overføringsfunktionen (del er fra 1.25)

For en inhomogen diff ligning:

Hvis {} er den stationære løsning

hvis og kun hvis {}.

Hvis {} er {} løsning til den tilsvarende homogene ligning og gættet
fejler. Gæt i stedet på {}.

\subsection{Systemer af
differentialligninger}\label{systemer-af-differentialligninger}

\paragraph{Form:}\label{form-1}

\subparagraph{Matrixform:}\label{matrixform}

\paragraph{Løsninger:}\label{luxf8sninger-1}

2.7 Egenværdimetoden (når am=1)

Et sæt relle løsninger til den homogene ligning {} bestemmes på flg.
måde\\
(a) For hver reel egenværdi, med tilhørende reel og egentlig egenvektor
{} opskrives løsningen

(b) Komplekse egenværdier optræder i par {}. Vælg en egentlig egenvektor
{} hørende til egenværdien {} og opskriv løsningerne

og

2.11 Løsninger ved multiplicitet {} (når am\textgreater1)

(a) For hver egenværdi {} med am {} findes vektorer {} så:

{}{}

(b) Antag, at {} er et par komplekst konjugerede egenværdier med am {}
og gm {}. Lad {}. Da findes vektorer {} således at

er lineært uafhængige reelle løsninger til systemet {}.

2.12 Fuldstændige reelle løsning efter 2.7 og 2.11

Den fuldstændige reelle løsning til det homogene
differentialligningssystem {} kan bestemmes på flg. måde:

\begin{enumerate}
\tightlist
\item
  For hver reel egenværdi {} opskrives løsningerne i 2.7(a)/2.11(a).
\item
  For hvert par {} af komplekst konjugerede egenværdier opskrives
  løsningerne i sætning 2.7(b)/2.11(b).
\end{enumerate}

Den fuldstændige reelle løsning til differentialligningssystemet fås ved
at danne linearkombinationer af de fundne {} løsninger, med reelle
koefficienter.

\paragraph{Fundamentalmatricen}\label{fundamentalmatricen}

2.14 Fundamentalmatricen

Til ethvert sæt af {} lin. uaf. løsninger {} til det hom. system {}
knyttes en {} fundamentalmatrix:

Søjlerne i {} består altså af vektorerne {}.

\paragraph{Overføringsfunktioner}\label{overfuxf8ringsfunktioner}

Overføringsfunktion for et system af diff. lign.

Betragt systemet\\
{}\strut \\
{}\strut \\
Da er

hvor

\paragraph{Stabilitet}\label{stabilitet}

2.28 {} Stabilitet (marginal stabilitet)

Det homogene system {} siges at være stabilt, hvis enhver løsning {} er
begrænset.\\
Dvs. at der findes en konstant {} så {}

2.38 Asymptotisk stabilitet

Det homogene system {} siges at være asymptotisk stabilit hvis og kun
hvis alle egenværdier for {} har negativ realdel.

2.41 Routh-Hurwitz kriterium

Givet et polynomium med reelle koefficienter, på formen

har alle rødder negativ realdel hvis og kun hvis de flg to kriterier er
opfyldt:

\begin{enumerate}
\tightlist
\item
  Alle koefficienterne er positive, dvs {} for {}.
\item
  Alle {} determinanter, {}, på formen

  er positive, dvs {}.
\end{enumerate}

\paragraph{Stabilitet for inhomogene
systemer}\label{stabilitet-for-inhomogene-systemer}

2.47 Asymptotisk stabilitet for inhomogene systemer

Et inhomogent system {} er asymptotisk stabilt hvis og kun hvis det
tilsvarende homogene system er asymptotisk stabilt.

2.48+2.49 BIBO-stabilitet

Det inhomogene system {} siges at være BIBO-stabilt hvis løsningerne {}
hørende til enhver begrænset påvirkning {} er begrænsede.\\
Det inhomogene system er BIBO-stabilt hvis og kun hvis det tilhørende
homogene system er asymptotisk stabilt.

\subsection{Talfølger}\label{talfuxf8lger}

\paragraph{Form:}\label{form-2}

\paragraph{Konvergens}\label{konvergens}

4.5 Konvergens for talfølger

En talfølge {} siges at være konvergent hvis der findes et tal {}
således at

Ellers er den divergent.

4.10 Regler for konvergens af to talrækker

Hvis {} og {} gælder

\begin{enumerate}
\item
\item
\item
  Hvis {} er {} for et vist trin og
\item
  Hvis {} og {} er {}
\end{enumerate}

\subsection{Uendelige rækker}\label{uendelige-ruxe6kker}

\paragraph{Form:}\label{form-3}

Den {}\textquotesingle te afsnitssum {}:

\paragraph{Konvergens generelt}\label{konvergens-generelt}

4.15 Definition på konvergens for uendelige rækker

Hvis talfølgen {} er konvergent med grænseværdien {} for {} siges {} at
være konvergent med sum {}, {}.

Hvis {} er divergent, er {} divergent og tillægges ingen værdi.

4.17 Konvergens for to uendelige rækker

Hvis {} og {} er konvergente er {} konvergent med sum {}.

\paragraph{Konvergenskriterier}\label{konvergenskriterier}

4.19 {}\textquotesingle te ledskriteriet

Hvis {} er konvergent gælder

Dette er ensbetydende med:\\
Hvis {} ikke går mod {} for {} er {} divergent.

4.20 Sammenligningskriteriet

For {}:

\begin{enumerate}
\tightlist
\item
  Hvis {} er konvergent er {} også konvergent.
\item
  Hvis {} er divergent er {} også divergent.
\end{enumerate}

4.24 Ækvivalentskriteriet

To rækker {} og {} med positive tal er ækvivalente hvis der findes en
konstant {} så

I så fald er begge rækker konvergente, eller begge rækker divergente.

4.30 Kvotientkriteriet

Antag {} for alle {} og der findes et {} så {} for {}.

\begin{enumerate}
\tightlist
\item
  Hvis {} er {} absolut konvergent
\item
  Hvis {} er {} divergent
\item
  Hvis {} kræves yderligere info.
\end{enumerate}

\paragraph{Absolut og betinget
konvergens}\label{absolut-og-betinget-konvergens}

4.26+4.27 Definition på absolut konvergens

{} er absolut konvergent hvis {} er konvergent.

Hvis {} er absolut konvergent er rækken konvergent.

4.28 Definition på betinget konvergens

{} er betinget konvergent hvis {} er konvergent og {} er divergent.

\paragraph{Summer}\label{summer}

4.33 Integralkriteriet

Lad {} være kontinuert og aftagende.\\
Betragt {}

\begin{enumerate}
\item
  Hvis {} er konvergent er {} konvergent og
\item
  Hvis {} er divergent er {} divergent.
\end{enumerate}

4.35 {} "Metode 1"

Antag {} er kontinuert og aftagende for {}.\\
Antag {} er konvergent. Da gælder

Vælg derfor {} således at

4.35 {} "Metode 2", mere præcis

Antag {} er kontinuert og aftagende for {}.\\
Antag {} er konvergent. Da gælder

Vælg derfor {} således at

4.38 Leibniz\textquotesingle{} kriterium

Antag {} (positive og aftagende) og {} for {}.\\
Da er {} konvergent og

\subsection{Uendelige rækker med variable
led}\label{uendelige-ruxe6kker-med-variable-led}

\paragraph{Form:}\label{form-4}

Kvotientrække:

Potensrække:

\paragraph{Konvergens og summer}\label{konvergens-og-summer}

5.2 Konvergens og sum for kvotientrække

En kvotientrække {} er konvergent hvis og kun hvis {}.\\
For {} er summen

For rækken {} er summen

5.13 Konvergensscenarier for potensrække

For en potensrække {} gælder ét af følgende:

\begin{enumerate}
\tightlist
\item
  Rækken er kun konvergent for {}
\item
  Rækken er absolut konvergent for alle {}
\item
  Der findes et tal {} således at rækken er absolut konvergent for {} og
  divergent for {}.
\end{enumerate}

\subparagraph{Funktioner som uendelige
rækker}\label{funktioner-som-uendelige-ruxe6kker}

5.7 Funktionsfremstilling som uendelig række

Antag at {} er vilkårligt ofte differentiabel, og at der findes en
konstant {} således at {}.\\
Lad {}. Så gælder at

5.17 Differention af potensrække

Givet en potensrække {}, med konvergensradius {} definer

Så er funktionen {} differentiabel, og

Generelt

\paragraph{Uniform konvergens}\label{uniform-konvergens}

5.28 Definition af uniform konvergens

Antag at {} er konvergent for {}, og sæt

Den uendelige række {} er uniform konvergent hvis der for ethvert {}
findes et {} således at

5.31 Definition af majorantrække

Lad funktionerne {} være definerede på et interval {} og betragt den
formelle række

\begin{enumerate}
\item
  En række {} med konstante og positive led {} er en majorantrække for
  rækken ovenfor på intervallet {}, hvis der for et hvert {} gælder at
\item
  En række {} der er konvergent og opfylder ovenstående ulighed siges at
  være en konvergent majorantrække for {}.
\end{enumerate}

5.33 Uniform konvergens pba. majorantrække

Antag at funktionerne {} er definerede på et interval {} og at rækken {}
har en konvergent majorantrække.\\
Så konvergerer den uendelige række {} uniformt på intervallet {}.

\subsection{Fourierrækker}\label{fourierruxe6kker}

\paragraph{Definitioner}\label{definitioner}

6.1 Fourierrækken

Til en {}-periode funktion {} knyttes en uendelig række, Fourierrækken:

hvor

6.5 Forskydning

Hvis {} er en {}-periodisk funktion er

6.6 Lige og ulige funktioner

Hvis {} er lige er {} og

Hvis {} er ulige er {} og

6.25 Fourierrækken på kompleks form

Ved Fourrierrækken for {} på kompleks form forstås

hvor

Vi har yderligere for {} at

\paragraph{Kontinuitet og
differentiabilitet}\label{kontinuitet-og-differentiabilitet}

Definition på stykkevis kontinuitet

Lad

Hvis {} er kontinuert i {} er {}.

\textbf{Stykkevis kontinuitet:}\\
{} ({}-periodisk) er stykkevis kontinuert hvis {} og {} findes for alle
{} og {} for alle {} pånær et endeligt antal {}.

6.13 Definition på stykkevis differentiabilitet

{} ({}-periodisk) er stykkevis differentiabel hvis den er stykkevis
kontinuert og der for ethvert {} findes to tal {} så

\paragraph{Fouriers sætning}\label{fouriers-suxe6tning}

6.16 Fouriers sætning

\begin{enumerate}
\item
  Hvis {} er en {}-periodisk stykkevis differentiabel funktion
  konvergerer Fourierrækken for alle {}.
\item
  Hvis {} er kontinuert i {} gælder
\item
  Hvis {} er diskontinuert i {} gælder

  æ
\end{enumerate}

\paragraph{Approximation via
fourierrækken}\label{approximation-via-fourierruxe6kken}

6.12 Konvergens for Fourierrækken

Hvis {} er stykkevis differentiabel, {}-periodisk og kontinuert, så
konvergerer fourierrækken mod {}.

6.21 Approximation via Fourierrækken

Antag at funktionen {} er {}-periodisk, kontinuert og stykkevis
differentiabel, med Fourierkoefficienter {}. Så gælder for ethvert {} at

Vælg derfor {} således at

\subsection{Fourierrækkemetoden}\label{fourierruxe6kkemetoden}

\paragraph{System:}\label{system}

Vi betragter differentialligningssystemet {}:

med overføringsfunktionen

hvor {}. Hvis {} er {} en partikulær løsning til {}, (stationær
løsning).

Lad ydermere {} være asymptotisk stabilt, dvs. for alle løsninger {} til
{} gælder {} for {} (ensbetydende med at alle egenværdier for {} har
negativ realdel).

\paragraph{Fourierrækkemetoden}\label{fourierruxe6kkemetoden-1}

7.7 Stationær løsning

Betragt systemet {}.\\
For enhver påvirkning på formen

fås løsningen

7.8 Fourierrækkemetoden

Betragt systemet {}.\\
Hvis {} er {}-periodisk, stykkevis differentiabel og kontinuert med {}
er det stationære svar

\paragraph{Parsevals sætning}\label{parsevals-suxe6tning}

6.30 Parsevals sætning

Hvis {} og

gælder

Metode: Relativ tolerence

Givet en relativ tolerence {}, {} bestem {}, så

\subsection{Potensrækkemetoden}\label{potensruxe6kkemetoden}

5.21 Identitetssætningen for potensrækker

Hvis koefficienterne {} for et passende {} opfylder at

så er {} for alle {}.

\end{document}
